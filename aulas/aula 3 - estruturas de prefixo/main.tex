\documentclass{beamer}
\usetheme{Boadilla}
\usepackage{tp} 
\usepackage{pgfplots}
\pgfplotsset{compat=1.18}
\usepackage{amsmath}
\title{Aula 3}
\subtitle{Estruturas de Prefixo}
\author{Maratona de Programação}
\institute{FEI}
\date{\today}
\setbeamertemplate{navigation symbols}{}

\begin{document}
\begin{frame}
	\titlepage
\end{frame}

\begin{frame}{Fundamento: queries estáticas de soma}
	\textcolor{structure}{Problema}: dado um vetor A de tamanho N $(N \leq 10^6,  \lvert A_i \rvert \leq 10^9)$ , responda Q queries $(Q \leq 10^6)$ do tipo:
	\begin{center}
		Dados dois índices $l$ e $r$ $(1 \leq l < r \leq N)$, \\
		computar a soma dos elementos do vetor $A$ de $l$ até $r$. \\
		Formalmente, respoder para cada query:
		\textcolor{structure}{
			\[
				\sum_{i=l}^{r} A_i = A_l + A_{l+1} + \dots + A_r
			\]
			}
	\end{center}
\end{frame}

\begin{frame}{Fundamento: queries estáticas de soma}
	\framesubtitle{Exemplos}

	\pause
	\begin{center}
		$l = 1, r = 5$
		\renewcommand{\arraystretch}{1.3}
		\setlength{\tabcolsep}{8pt}

		\vspace{10pt}
		\begin{tabular}{c c}
			$A:$ &
			{
				\rowcolors{1}{structure!10!white}{structure!5!white}
				\begingroup
				\color{structure}
				\begin{tabular}{|c|c|c|c|c|c|c|c|c|}
					\hline
					\textbf{7} & \textbf{5} & \textbf{2} & \textbf{3} & \textbf{2} & 7 & 8 & 5 & 1 \\
					\hline
				\end{tabular}
				\endgroup
			}
		\end{tabular}

		\vspace{-8pt}
		\begin{tabular}{c c}
			\hspace{25pt}
			\setlength{\tabcolsep}{9.3pt}
			{\tiny
			\begin{tabular}{ccccccccc}
				1 & 2 & 3 & 4 & 5 & 6 & 7 & 8 & 9 \\
			\end{tabular}
			}
		\end{tabular}
	\end{center}

	\pause
	\begin{center}
		$l = 5, r = 8$
		\renewcommand{\arraystretch}{1.3}
		\setlength{\tabcolsep}{8pt}

		\vspace{10pt}
		\begin{tabular}{c c}
			$A:$ &
			{
				\rowcolors{1}{structure!10!white}{structure!5!white}
				\begingroup
				\color{structure}
				\begin{tabular}{|c|c|c|c|c|c|c|c|c|}
					\hline
					7 & 5 & 2 & 3 & \textbf{2} & \textbf{7} & \textbf{8} & \textbf{5} & 1 \\
					\hline
				\end{tabular}
				\endgroup
			}
		\end{tabular}

		\vspace{-8pt}
		\begin{tabular}{c c}
			\hspace{25pt}
			\setlength{\tabcolsep}{9.3pt}
			{\tiny
			\begin{tabular}{ccccccccc}
				1 & 2 & 3 & 4 & 5 & 6 & 7 & 8 & 9 \\
			\end{tabular}
			}
		\end{tabular}
	\end{center}

	\pause
	\begin{center}
		$l = 4, r = 6$
		\renewcommand{\arraystretch}{1.3}
		\setlength{\tabcolsep}{8pt}

		\vspace{10pt}
		\begin{tabular}{c c}
			$A:$ &
			{
				\rowcolors{1}{structure!10!white}{structure!5!white}
				\begingroup
				\color{structure}
				\begin{tabular}{|c|c|c|c|c|c|c|c|c|}
					\hline
					7 & 5 & 2 & \textbf{3} & \textbf{2} & \textbf{7} & 8 & 5 & 1 \\
					\hline
				\end{tabular}
				\endgroup
			}
		\end{tabular}

		\vspace{-8pt}
		\begin{tabular}{c c}
			\hspace{25pt}
			\setlength{\tabcolsep}{9.3pt}
			{\tiny
			\begin{tabular}{ccccccccc}
				1 & 2 & 3 & 4 & 5 & 6 & 7 & 8 & 9 \\
			\end{tabular}
			}
		\end{tabular}
	\end{center}

\end{frame}

\begin{frame}{Fundamento: queries estáticas de soma}
	\framesubtitle{Ideia de solução 1: brutar cada query}

	\begin{itemize}
			\pause
		\item \textcolor{structure}{Ideia:} Para cada query, iterar de $l$ até $r$ no vetor e ir acumulando $A_i$.
			\pause
		\item Isso com certeza da o resultado certo, mas \dots
			\pause
		\item Por que essa solução \textcolor{red}{\textbf{não}} é ótima?
			\pause
		\item Complexidade de tempo: \textbf{$\mathcal{O}{(N*Q)}$}.
			\pause
		\item Quantidade de operacoes no pior caso: \textbf{$10^6 * 10^6 = 10^1^2$}.
			\pause
		\item Resultado: \textcolor{red}{\textbf{Time Limit Exceeded (TLE)}}.
	\end{itemize}

\end{frame}

\begin{frame}{Fundamento: queries estáticas de soma}
	\framesubtitle{Ideia de solução 2: pré-computar todas as possíveis queries}

	\begin{itemize}
			\pause
		\item \textcolor{structure}{Ideia:} Antes das queries, criar um map ou um vetor bidimensional que guarda a resposta pra query $(l, r)$.
			\pause
		\item preencher esse vetor com a resposta de cada uma dessa queries.
			\pause
		\item Por que essa solução \textcolor{red}{\textbf{não}} é ótima?
			\pause
		\item Quantas queries \textbf{distintas} podem ter?
			\pause
			\begin{itemize}
				\item Melhor perguntando: quantos pares ordenados $(l, r)$, com $l$ e $r$ $(1 \leq l < r \leq N)$ existem?
					\pause
				\item $\frac{N * (N - 1)}{2}$ (uma quantidade quadrática).

			\end{itemize}
			\pause
		\item Complexidade de tempo e espaço: \textbf{$\mathcal{O}{(N^2)}$}.
			\pause
		\item Quantidade de operacoes no pior caso: \textbf{$10^6 * 10^6 = 10^1^2$}.
			\pause
		\item Resultado: \textcolor{red}{\textbf{Time Limit Exceeded (TLE) ou MLE}}.
	\end{itemize}

\end{frame}

\begin{frame}{Fundamento: queries estáticas de soma}
	\framesubtitle{Ideia de solução 3: Soma de Prefixo}


	\begin{itemize}
			\pause
		\item \textcolor{structure}{Ideia:} Para cada query, buscamos o primeiro $i$ tal que $A_i \geq X$.
			\pause
		\item Começamos considerando todo o segmento \([1, N]\) como possível resposta.
			\pause
		\item Escolhemos o valor no meio do segmento e verificamos se ele é \(\geq X\).
			\pause
		\item Se for, ele pode ser uma resposta, mas ainda tentamos encontrar uma posição melhor à esquerda, olhando agora para $[1, meio]$.
			\pause
		\item Se não for, descartamos ele e toda a parte à esquerda, olhando agora para $(meio, N]$.
			\pause
		\item Repetimos até o sobrar só um valor. A posição final encontrada (se houver) é a menor com valor \(\geq X\).
	\end{itemize} 
\end{frame}



\end{document}
